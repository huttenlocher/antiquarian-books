\documentclass[a5paper,openany]{book}
\setlength{\pdfpagewidth}{148mm}
\setlength{\pdfpageheight}{210mm}

\usepackage{fancyhdr}
\usepackage[T1]{fontenc}
\usepackage[utf8]{inputenc}
\usepackage[ngerman]{babel}
\usepackage[margin=14.9mm, paperwidth=148mm, paperheight=210mm]{geometry}
\usepackage{indentfirst}
\usepackage{titlesec}
\usepackage{csquotes}
\usepackage{qrcode}
\usepackage[perpage]{footmisc}

\titleformat{\chapter}[display]
{\bfseries\LARGE}
{\filright}
{1ex}{\thechapter. }[]

\MakeOuterQuote{"}

\pagestyle{fancy}
\renewcommand{\headrulewidth}{0pt}
\fancyhead{}

\title{Habt acht auf euch selbst!}
\author{Johannes Schneider}

\makeatletter
\newcommand\customtitle{%
  \begin{titlepage}
    \null\vfil
    \begin{center}
      {\Huge \textbf{\@title} \par}
      \vskip 2.5em
      {\LARGE Ein brüderliches Wort an Seelsorger \par}
      \vskip 1em
      {\large von \par}
      \vskip 1em
      {\Large \@author \par}
    \end{center}\par
    \vfil\null
  \end{titlepage}
}
\makeatother

\begin{document}
\customtitle

\newpage
\thispagestyle{empty}
\section*{Anmerkungen zur Neuausgabe}
Das vorliegende Buch habe ich antiquarisch erworben und mit großem Gewinn gelesen. Da eine Neuauflage wohl nicht zu erwarten und selbst mein eigenes Druckexemplar (4. Auflage, Brunnen-Verlag Gießen und Basel, Druckerei Hermann Rathmann Marburg) noch in alter, schwer lesbarer Frakturschrift gesetzt ist, habe ich den Text mithilfe von \LaTeX{} neu erfasst, um das Buch einem breiteren Leserkreis wieder zugänglich zu machen. Außerdem habe ich den Text an die neue deutsche Rechtschreibung angepasst und für Bibelzitate, die der Autor ohne Quellenangaben verwendet, die entsprechenden Schriftstellen in den Fußnoten vermerkt. Schriftstellenangaben im Text selbst stammen hingegen vom Autor und sind so in der Originalausgabe enthalten. Das Format der Schriftstellenangaben ist durchgehend angelehnt an die revidierte Lutherbibel 2017 (Deutsche Bibelgesellschaft). Die Aufteilung in Kapitel und Absätze entspricht exakt dem Original, ebenso alle Hervorhebungen und Zwischenüberschriften. Lediglich das Wort \textsc{Herr} habe ich an allen Stellen zusätzlich hervorgehoben, um zu verdeutlichen, dass es sich um den lebendigen Gott handelt. Die gelegentlichen Bezugnahmen des Autors auf Psychologie und Katholizismus mögen inzwischen bedenklich erscheinen, aus Respekt vor dem Autor sind sie unverändert beibehalten. Der Leser ist hier aufgerufen, Weisheit und geistliches Unterscheidungsvermögen zu üben!
\par
Im Übrigen wünsche ich von Herzen, dass die Lektüre dieses Buches – seinem Titel gemäß – allen Lesern dazu verhilft, dem apostolischen Befehl gründlich Gehorsam zu leisten. Der auferstandene Heiland stehe uns bei!
\begin{flushright}
Euer Bruder in Christus,\\
Anton Huttenlocher\\
Berlin, 4. Sonntag nach Trinitatis 2024
\end{flushright}
\vspace*{\fill}
P.S. Einen Fehler entdeckt? Erstelle ein Ticket auf GitHub!
\par
\begin{flushright}
\qrcode[height=1in]{https://github.com/huttenlocher/antiquarian-books/issues}
\end{flushright}
\newpage
\setcounter{page}{1}
Das Thema, das wir in brüderlicher Freimütigkeit und in ernster Selbstprüfung vor dem \textsc{Herrn} besprechen wollen, führt uns hinein ins Quellgebiet unseres Christenlebens und unseres Dienstes. In der heiligen Sorge um unsere eigene Seele liegt das Geheimnis der Fruchtbarkeit unserer Seelsorge an den uns anvertrauten Seelen. Im Alten Bund war das Dienstfeld der Priester in zwei Gebiete eingeteilt (4.Mose 18,7): Der Dienst \emph{vor dem Vorhang}, die Geschäfte des Altars, und der Dienst \emph{inwendig, hinter dem Vorhang}, das Verharren im Heiligtum. Die Segenswirkungen des Dienstes vor dem Vorhang hingen ab von der Treue und Sorgfalt der Pflege des Heiligtums hinter dem Vorhang. Ebenso werden auch bei uns die Geschäfte des Altars, die Heilsverkündigung und das priesterlich-seelsorgerliche Wirken an den Seelen, die der \textsc{Herr} uns anvertraut hat, nur in dem Maß von Ewigkeitskraft durchwirkt sein und Ewigkeitswert haben, als auch wir treu sind im Innendienst, treu in der Sorge um unsere eigene Seele. Ja, wir wollen es uns immer aufs Neue klar bewusst werden, dass wir den anvertrauten Seelen nicht mehr geben können, als wir selber besitzen, und dass wir ihnen nicht mehr dienen können, als Jesus uns dienen kann. Unsere eigene Seele ist die erste Seele, die Gott uns unvertraut hat. Ihr soll die erste Sorge des Seelsorgers gelten.
\par
Liebe Brüder, geschieht dies bei uns allen? Sind wir im Innersten uns selbst gegenüber treu, ganz treu? Ist wirklich der Dienst hinter dem Vorhang täglich unsere erste und heiligste Sorge, von der wir uns durch keine andere Pflicht abhalten lassen? Wenn unsere Seelen plötzlich ihre körperliche Umhüllung abstreifen und in Sichtbarkeit treten könnten, würden sich jetzt lauter Seelen offenbaren, die gereinigt sind durch das Blut Jesu Christi, die normal wachsen in allen Stücken zu Christus hin, die gesättigt sind von göttlichen Gedankenkräften, Seelen, die die Merkmale einer treuen, glaubensmutigen, fortlaufenden Heiligungsarbeit an sich tragen? Ich werfe diese Gewissensfrage auf als einer, der sich in ernster Selbstprüfung tief beugen musste vor seinem \textsc{Herrn} in der Erkenntnis so vieler Untreuen und Vernachlässigungen der eigenen Seele gegenüber, und der sich aufs Neue dessen bewusst geworden ist, dass unsere erste Pflicht den Seelen gegenüber, denen wir dienen dürfen, die Sorge um unsere eigene Seele ist, die Durchheiligung unserer Persönlichkeit. Nehmt also meine Ausführungen an als das, was sie sind: ein brüderliches Selbstbekenntnis!
\begin{center}
* * *
\end{center}
\par
Wir wollen uns zunächst die Notwendigkeit und Dringlichkeit der Seelsorge an Seelsorgern wieder einmal klar vergegenwärtigen und dann die verschiedenen Mittel und Wege ins Auge fassen, die uns zu ihrer Ausübung zur Verfügung stehen.
\renewcommand\thechapter{\Roman{chapter}}
\setcounter{section}{1}
\chapter{Die Notwendigkeit und Dringlichkeit der Seelsorge an Seelsorgern.}
Wir könnten zu ihrer Begründung etwa hinweisen auf die Größe, Schwierigkeit und außerordentliche Wichtigkeit der Aufgaben, die uns anvertraut sind, auf die Verantwortung, die auf uns liegt, auf die furchtbaren Gegenmächte, mit denen wir es zu tun haben, wenn wir unseren Seelsorgerdienst treu erfüllen wollen. Ich möchte mich aber nur auf eins beschränken, nämlich kurz hinweisen auf die Gefahren, die gerade durch die Ausübung unseres Dienstes unserer eigenen Seelsorgerseele drohen. Es wird ja nicht jeder von uns diesen Gefahren in gleicher Weise ausgesetzt sein. Da spielt die Naturanlage des Einzelnen eine große Rolle, sein Temperament, seine Erziehung und Bildung, die Verhältnisse, in denen er steht, ja auch sein Körperzustand. Aber mehr oder weniger werden wir sie doch alle kennen, die fünf Seelsorgergefahren, die uns beständig auf unseren Dienstwegen drohen.
\par
Wir nennen als erste \emph{die Gewöhnung}.
\par
Es ist ja eine ganz allgemeine psychologische Tatsache, dass die Seele durch genügend lange Einwirkung eines bestimmten Eindrucks sich an alles, ans Höchste wie ans Tiefste, ans Schönste wie ans Hässlichste, ans Heiligste wie ans Gemeinste so gewöhnen kann, dass sie im Innersten nicht mehr darauf reagiert. Äußerlich mag der Einfluss noch fortbestehen, aber die innerste Anteilnahme der Seele hört auf. So entsteht die Gewöhnung, der Leerlauf, die Routine. Auch das Lebendigste und Heiligste kann mechanisiert werden. Schon im Alten Bund sollten die vielen Warnungen in den Kultusgesetzen, die äußerst scharfen Strafandrohungen bei der geringsten Dienstverletzung beständige Weckrufe sein für die Priester und Leviten, doch ja nicht an das Hellige sich zu gewöhnen, ja nicht den heiligen Dienst mit unheiligen Händen zu verrichten. Und wenn im Neuen Bund Paulus seinem Knecht Timotheus zuruft: "Ich erinnere dich daran, dass du neu anfachest die Gabe Gottes, die in dir ist durch die Auflegung meiner Hände", so ist damit bezeugt, dass man durch Handauflegung auch zum heiligen Dienst verordnet worden sein kann und diesen Dienst vielleicht äußerlich mit großem Eifer treibt, und dass dabei innerlich die Gabe Gottes, das heilige Feuer, immer schwächer klimmen und allmählich verlöschen kann. Ja, gerade wir Brüder, die wir beständig im Heiligtum stehen und die heiligen Geräte gebrauchen dürfen, wir, die wir so viel in Gottes Wort forschen und darüber reden, so viel mit den Seelenanliegen unserer Gemeindeglieder zu tun haben, gerade wir stehen ganz besonders in Gefahr, vielleicht unbewusst in ein gewohnheits- und programmmäßiges Wirken zu geraten, geistliche Handwerker zu werden. Es wird vielleicht noch lange weiter gepredigt und gearbeitet; aber es fehlt dem Dienst die Vollmacht Jesu, es fehlt ihm die verborgene Triebkraft des heiligen Feuers. Die Liebe erkaltet allmählich, das Verantwortlichkeitsbewusstsein erlahmt; wir sind in Gefahr, ein tönendes Erz und eine klingende Schelle zu werden. Ich höre immer noch Inspektor Rappard den jungen Brüdern vor der Einsegnung zurufen: "Brüder, werdet keine geschobenen Menschen!" Das heißt, werdet keine Reichsgottesarbeiter, die nur von der Arbeit geschoben werden, bei denen aber das Geisteswirken innerlich nicht mehr Schritt hält! O, seien wir mißtrauisch gegen uns selbst, wenn uns die Wortverkündigung, das öffentliche Beten, das Reden mit den Seelen zu leicht geht und wir keine Seelennot und Arbeit dabei empfinden, "wenn's läuft", wie man sagt. Prüfen wir dann genau nach, ob dieses Laufen nicht am Ende "Leerlauf" ist! Wehe uns, wenn die Gewöhnung bei uns Einkehr halten kann! Dann folgen ihr auf dem Fuß ihre vier Schwestern: die Verflachung, die Heuchelei, die Unfruchtbarkeit, die Untauglichkeit. Hüte deine Seele vor Gewöhnung!
\par
Das führt uns zu einer zweiten Gefahr, die oft hinter der Gewöhnung steckt als eine ihrer Ursachen. Wir meinen den \emph{religiösen Großbetrieb}, die Vielgeschäftigkeit, die Gefahr, ganz aufzugehen im Dienst vor dem Vorhang, in den Geschäften des Altars. Gewiss ist ein verzehrender Eifer in der Arbeit für den \textsc{Herrn} an sich etwas Schönes. Aber es gilt doch, seine treibenden Motive zu prüfen. Bei dem einen mag es wirklich heiliger Liebeseifer sein im Bewusstsein: Die Sache des Königs eilt. Bei andern mögen auch gewisse Temperamentsanlagen mitspielen. Sie leben überhaupt lieber in einem aufregenden Vielerlei wechselnder Arbeit als beim stillen Forschen im Studierzimmer. Es liegt ihnen das mehr. Noch andere sehen keine andere Möglichkeit. Die Arbeit ist eben da, häuft sich immer mehr, drängt sich immer gebieterischer auf und nimmt sie voll und ganz in Anspruch. Mögen die Gründe sein wie sie wollen, eins ist gewiss: Der Großbetrieb im Reichgottesdienst ist eine moderne Gefahr, die überaus verhängnisvoll sein kann.
\par
Wo so gearbeitet wird, arbeiten immer mehr die Nerven anstelle des Heiligen Geistes, die Ichkraft verdrängt die Jesuskraft. Man merkt dieses innere Verarmen vielleicht lange nicht. Man ist vielleicht ein geschätzter, geliebter, gesuchter Prediger; aber in Wirklichkeit ist all das, was auf solche Weise erreicht wird, nur Erfolg und nicht Frucht, nur Zuwachs und nicht Wachstum, und plötzlich gibt es einen Zusammenbruch beim Prediger selbst und in der Arbeit. Auch sind wir bei solcher Art der Arbeit in Gefahr, uns zu verlieren an die Menschen und an die Arbeit. Wir arbeiten nicht mehr rezeptiv, sondern nur noch produktiv. Die Folge davon ist, dass das gesammelte Geistes- und Gedankenkapital bald erschöpft ist. Die Selbsttätigkeit der Seele erlahmt, das eigene Denken verarmt und versiegt, und dann denken wir nur noch mit dem Geist und den Gedanken anderer. Wir verlieren unser Eigengepräge, das Gott uns als Anlage gegeben hat; wir werden ein allgemeiner Charakter, ein Produkt der Umwelt. Und doch, liebe Brüder, muss man sich selbst besitzen, um sich Gott und den Nächsten hingeben zu können.
\par
Hüten wir uns aufs Äußerste vor dem Großbetrieb! Verwechseln wir nicht Lebhaftigkeit mit Lebendigkeit, nicht Beweglichkeit mit Bewegung! Unsere Gemeinschaftsbeweglichkeit mit Bewegung! Unsere Gemeinschaftsbewegung soll unbedingt eine geistgewirkte, lebendige Bewegung bleiben; aber Bewegung bleibt sie nicht durch äußere Beweglichkeit und Großbetrieb, sondern nur durch das Wehen des Heiligen Geistes, durch sein sanftes, stilles Säuseln. Beherzigen wir immer wieder das Wort, das Adolphe Monod, der reichgesegnete Zeuge in Frankreich, auf seinem Sterbebette einem Freund zurief, der ihn fragte, was er tun würde, wenn er noch einmal sein Leben beginnen könnte: "Ich würde weniger arbeiten und mehr beten".
\par
Eine dritte Gefahr, die aufs Engste mit der eben genannten zusammenhängt, ist die \emph{fromme Selbsttäuschung}. Dadurch, dass unsere Mitmenschen es als ganz selbstverständlich betrachten, dass wir das sind, was wir predigen, und als solche uns behandeln, – dadurch, dass so manche uns Weihrauch streuen und so wenige uns ehrlich die Wahrheit sagen, – dadurch, dass wir uns in Gedanken beständig mit den höchsten Idealen der Heiligung beschäftigen und viel darüber reden, sind wir in Gefahr, uns in einer höheren Lebenssphäre zu wähnen, als wir vielleicht in Wirklichkeit sind. Wir glauben, manches von dem, was wir mit Begeisterung verkündigen, selbstverständlich zu besitzen, während wir es in Wirklichkeit noch nicht im Innersten durchlebt haben und es noch nicht unser innerster Besitz ist. Die Selbsttäuschung ist unsere größte Gefahr. Sie verbindet sich dann gewöhnlich mit Eigensinn und Hochmut, und das ist der Weg, auf dem man ganz unvermerkt in die Übergeistlichkeit hinübergleitet. Von der Übergeistlichkeit ist oft nur noch ein Schritt hinab in die grobe Fleischlichkeit. Und darum, Brüder, machen wir es täglich wie die Eisenbahnangestellten, die immer wieder nach der Einfahrt eines Zuges mit einem Eisenhammer an die Räder schlagen, um zu prüfen, ob der Klang noch rein ist, ob keine Risse entstanden sind. Prüfen wir auch täglich nach, auf welchen Motiven unser Dienst rollt, ob die Wärme unseres Eifers wirklich heilige Liebeswärme ist, ob unser Arbeitstrieb wirklich heiliger Rettertrieb ist!
\par
Eine vierte Gefahr ist die \emph{geistliche} und \emph{geistige Unterernährung}. Unsere Seele gibt mehr aus, als sie einnimmt; der Kräfteverbrauch ist größer als die Kräftezufuhr. Dass das nicht ohne verhängnisvolle Wirkung für unser Seelenleben und unseren Dienst bleiben kann, ist ja selbstverständlich. So entsteht der merkwürdige Zustand, den wir wohl am besten seelische Gichtbrüchigkeit nennen könnten, ein allmähliches seelisches und geistiges Erlahmen und Erstarren. Die Gicht ist ja bekanntlich die Folge von schlechter Säftemischung und von Erkältung. So ist es auch im Seelenleben. Wenn einer Seele, die täglich unter dem Druck starker Anforderungen steht, nicht genügend Geisteskräfte zugeführt werden, oder wenn diese Kräfte nicht genügend innerlich verarbeitet, nicht recht vermischt werden mit den vorhandenen Kräften, und wenn dazu noch äußere Kälte herrscht, keine geistige Anregung, keine Gelegenheit zu tiefer, warmer Gemeinschaft mit verwandten Seelen vorhanden ist, dann muss eine Seele naturgemäß nach einer gewissen Zeit geistlich und geistig verarmen, degenerieren, gichtbrüchig werden. Drummond hat durchaus recht, wenn er redet von Naturgesetzen in der Geisteswelt. Wie in der Natur ein gesundes Wachstum ohne Ernährung der Zellgewebe unmöglich ist, so ist auch in unserem Innenleben und in unserem Dienst ein wirklich gesegnetes Wachstum, eine wirkliche Fruchtbarkeit nicht möglich, wenn die Seele an geistlicher und geistiger Unterernährung leiden muss. Und ach, wie manche unter uns spüren sehr wohl die Gefahr und ihre Wirkungen, leiden sehr darunter, sehen aber keine Abhilfe. Wir werden noch darauf zurückkommen.
\par
Und endlich lasst mich noch auf eine fünfte Gefahr hinweisen, die vielfach in einem inneren psychologischen Zusammenhang steht mit den obengenannten. Wir meinen die \emph{Verzagtheit}, die Seelenermattung und Dienstmüdigkeit. Die meisten unter uns kennen sie wohl aus eigenster Erfahrung, diese merkwürdige, geheimnisvolle Macht, die oft plötzlich wie Meereswogen über Herz und Gemüt schlagen kann, alles in uns lahm legt, die Freude, die Hoffnung, die Zuversicht, das Gebet, den Lebenswillen, und die auch über die stärksten und aufrechtesten Naturen kommen kann, wie über einen Josua und einen Elias. Wir wissen auch, wie unfähig zum Dienst die Verzagtheit uns machen kann, und wie auch auf diesem Gebiet das Wort gilt: "Draußen sind die Verzagten"\footnote{Offb 21,8+22,15}. Woher kommt die Verzagtheit? Sie kann ja sehr verschiedene Ursachen haben. Sie kann rein körperlich bedingt sein durch Herzaffektion und andere Leiden. Meistens aber liegen ihre Ursachen in der Seele selbst. Entweder ist sie die Folge seelischer Überspannung bei seelischer Unterernährung. Plötzlich gibt es einen Zusammenbruch. Oder die Seele ist durch Nöte und Aufgaben, durch Bekenntnisse aller Art überlastet worden, und man hat die rechte Entlastung im Gebet vor dem Hohenpriester versäumt. Auf einmal erschlafft die Tragkraft. Oder wir sind in tiefe Selbsterkenntnis geführt worden, in die Erkenntnis unserer Sündhaftigkeit und Unwürdigkeit und Unfähigkeit. Aber es hat unserer Selbsterkenntnis an der Einfalt des demütigen, mutigen Glaubens an den Heiland und Erlöser gefehlt, und plötzlich schlug die Beugung und Buße um in Verzagtheit. Oder – wir wollen es offen bekennen – manchmal ist die Verzagtheit nur ein verkappter Hochmut, eine gekränkte Eigenliebe. Ja, wie in unserem Leibesleben die körperliche Ermüdung nach dem Urteil der Physiologen eine stille Vergiftung ist, so kann die Seelenermüdung, die Verzagtheit auch manchmal Folge oder Äußerung einer stillen, verborgenen Seelenvergiftung sein. Es sind sündliche Giftstoffe in die Gedanken- und Gefühlswelt eingedrungen, von denen man sich nicht sofort im Gebet, durch das Blut Jesu hat reinigen lassen.
\par
Brüder, hüten wir uns vor der Verzagtheit, ganz besonders, wenn sie fromme Klagetöne anschlägt! Dann ist sie am gefährlichsten. Wie mancher Bruder ist dadurch dienstuntauglich geworden. Verzagtheit ist Sünde.
\par
Ich habe nun auf einige Gefahren hingewiesen, die gerade in und durch die Ausübung unseres herrlichen Dienstes unserer Seelsorgerseele drohen: Gewöhnung, Großbetrieb, Selbsttäuschung, Unterernährung, Verzagtheit. Wollen wir in unserem persönlichen Innenleben normal, gesund, natürlich, heilig wachsen in allen Stücken zu Christus hin, wollen wir dem hohen Ernst der Gegenwarts- und Zukunftsaufgaben, die an uns herantreten, gewachsen sein und sie zur Verherrlichung unseres hochgelobten Heilandes erfüllen, dann müssen wir diese Gefahren klar bewusst im Auge behalten und müssen frei werden und frei bleiben von ihren Umklammerungen. Zu dieser Überwinderstellung gelangen wir nur durch treue und gewissenhafte Seelsorge an unserer eigenen Seele. Das führt uns zum zweiten Teil unserer Ausführungen.
\chapter{Die Mittel und Wege der Seelsorge an Seelsorgern.}
Auch da wollen wir wieder zwei Gebiete unterscheiden: Die \emph{Selbstseelsorge} und die \emph{brüderliche Seelsorge}, mit andern Worten: Die Selbstseelsorge, die wir selbst unserer Seele erweisen können, und die brüderliche Seelsorge, die andere unserer Seele erzeigen.
\renewcommand\thesection{\Alph{section}.}
\section{Die Selbstseelsorge.}
Da möchten wir das feine Wort eines katholischen Seelsorgers, des Prälaten Jakob Schmitt, voranstellen: "Vergiss nicht, dass zuerst und zuletzt und zunächst und zuinnerst Gott selbst dein Seelsorger ist; aber er will dich selbst in der Sorge um deine Seele zum Gehilfen haben!" Wir dürfen also nicht in einer gewissen frommen Bequemlichkeit unsere Seele einfach Gott anvertrauen und ihm alles Weitere überlassen, ohne uns darum zu kümmern, sondern seine Seelsorge erfordert unsere intensive Anteilnahme. Es gilt, die Mittel und Wege, die er uns zur Verfügung stellt, treu zu benutzen. Lasst mich nur auf einige dieser Mittel und Wege der Selbstseelsorge hinweisen!
\par
Als Grundbedingung gesegneter Selbstseelsorge nennen wir \emph{die Stille}. "Nur an einer stillen Stelle legt Gott seinen Anker an", heißt es in einem Liede. Wir denken jetzt allerdings bei diesem Wort "Stille" zunächst nicht an eine Örtlichkeit, sondern vor allem an einen Zustand, an die innere Stille der Seele, an die innere Sammlung vor dem \textsc{Herrn}, an die Absonderung der Seele von der Außenwelt und ihre Verinnerlichung in die innerste Einsamkeit, in die Gemeinschaft mit Gott. Es handelt sich da nicht etwa um eine mystische Versenkung in eine überirdische Gefühlslosigkeit, sondern um das, was der Apostel "das  verborgene Leben mit Christo in Gott"\footnote{Kol 3,3} nennt. Wenn wir diese stille Sammlung und heilige Einsamkeit nicht vor allem in uns selbst tragen, werden wir sie sonst nirgends finden, auch nicht in der stillsten Bergeinsamkeit. Wir sollten sie in uns haben und genießen können, auch mitten im Gedränge der Menschen und der Arbeit, unabhängig vom Äußerlichen, wie ein François de Gales es bezeugen konnte: "Ich bin umgeben von Menschen; aber mein Herz ist einsam und still in Gott". Es muss etwas sich lagern zwischen unserer Seele und der Umwelt, wie die Wolkensäule zwischen dem Volk Israel und den nachjagenden Ägyptern sich lagerte. Und dieses Etwas, das Stille in unsere Seele bringt, ist die Gegenwart Gottes selbst. Wo die Herrlichkeit Jesu das Herz erfüllt, wo sein Geist, der Atem aus der ewigen Stille, den Seelengrund sanft durchwehen kann, da ist Stille, tiefe, starke, heilige, lebensvolle Stille, Heiligtumsstille. Da wird alles Irdische und Menschliche verdrängt, und die Seele genießt die unmittelbare Gemeinschaft mit Ihrem Bräutigam. Wo aber diese reale Innewohnung Jesu fehlt, da ist das Herz ein öffentlicher Platz, in den alle Straßen münden und alles einläuft.
\par
Brüder, sollte es denn uns weniger möglich sein, allezeit die Gegenwart Jesu in uns zu tragen, als ein Weltmensch dies mit seinen Götzen tun kann? Dass wir nicht auf einmal mühelos zur Seelenstille gelangen, sondern dass sie ein ernstes Gebets- und Willensringen erfordert, das bezeugt uns ja der Apostel selbst mit dem Zuruf: "Ringet darnach, dass ihr stille werdet!"\footnote{Hebr 4,11} Aber dass er uns zum Ringen auffordert, ist ein Beweis, dass diese Stille errungen werden kann. Und gerade wir, die wir so selten stille Sonntage haben, sollen vor allem ringen um die heilige Stille. Erst wenn diese Vorbedingung aller Selbstseelsorge vorhanden ist, können überhaupt die anderen Mittel und Wege ihre Wirkung ausüben.
\par
Durch diesen nachdrücklichen Hinweis auf die Bedeutung der inneren Stille wollen wir selbstverständlich nicht sagen, dass die äußere Stille und Einsamkeit überflüssig sei, im Gegenteil. Wenn sogar Jesus manchmal von allen und allem wegfloh in die Wüsteneinsamkeit, wie viel nötiger haben wir solche Stunden. Es ist nicht richtig, wenn etwa vielbeschäftigte Reichgottesarbeiter auf die äußere Stille glauben verzichten zu können und sich gar rühmen, soundso viele Jahre keine Ferien gehabt zu haben, etwa mit der Begründung, unser ganzes Leben solle eine anhaltende Sammlung sein. Soll denn nicht auch jeder Tag der Woche ein Tag des \textsc{Herrn} sein? Und doch hat Gott noch einen Tag in besonderer Weise dazu geheiligt. Soll nicht die ganze Welt Gottes Tempel sein? Und doch haben wir besondere Gotteshäuser zur stillen Andacht. Weil wir schwach sind, brauchen wir auch äußere Hilfsmittel. "Zerschlage nicht die Stufen der Treppe in der Meinung, größere Schritte machen zu können und rascher droben zu sein! Wir brauchen die Stufen, um höher zu gelangen." Wir brauchen auch Zeiten äußerer Stille und Sammlung, um innerlich stiller zu werden und zu wachsen.
\par
Wie können wir in der inneren und äußeren Stille und Sammlung Selbstseelsorge üben?
\par
Da nennen wir als Erstes \emph{die Selbstprüfung}. Das Gericht muss anfangen am Hause Gottes.\footnote{1.Petr 4,17} Ehrliche  Selbstprüfung ist die erste Bedingung, um den fünf Gefahren begegnen zu können, die wir vorhin nannten. Wir verstehen unter Selbstprüfung nicht Selbstbeschaulichkeit, auch nicht das, was die Psychoanalyse tut: Selbstzergliederung des Seelenlebens im Licht des eigenen Verstandes, sondern wir verstehen unter Selbstprüfung das Stehenbleiben hinter dem Vorhang im Lichtkreis der Heiligkeit Jesu, wie sie uns vom Kreuz am durchdringendsten entgegenstrahlt, das stille, sorgfältige Achten auf das, was der Heilige Geist uns aufdeckt. Er will uns ja in alle Wahrheit leiten, vor allem in Bezug auf uns selbst. Aber soll die Selbstprüfung von Segenswirkung sein, dann müssen wir eben in der Stille nachdenken, müssen uns selbst intensiv daran beteiligen. Das kann am besten geschehen, wenn wir unserer Seele ganz direkte Fragen vorlegen und sie schonungslos vor dem \textsc{Herrn} beantworten, etwa Fragen wie die: Worin werde ich noch am häufigsten versucht? "Die Höhenlage der Dinge, die uns bestürmen, anfechten, sind ein Gradmesser für unsere eigene innere Höhenlage." An der Art, am Inhalt, an der Wucht können wir merken, was für Zündstoff noch in uns vorhanden ist, wo unsere schwachen Punkte sind. Oder andere Fragen: Wie steht es mit meinem Gebetsleben, mit der Treue in der Fürbitte? Wie steht's mit dem Gebrauch meiner Zeit und Kraft, meiner Mittel? Wäre ich bereit, heute Nacht für meine Überzeugung zu sterben? Wenn nein, warum nicht? Was würde mich noch zurückhalten? Habe ich wirklich nur die Ehre Gottes, den Bau seines Reiches im Auge? Tue ich, was ich kann in der Vorbereitung auf meine Predigten? Wie steht es mit der Innenseite meines Privatlebens, meines Ehelebens, meines Familienlebens?
\par
Ja, Brüder, solch zentral gerichtete Gewissensfragen – wenn wir sie ehrlich beantworten und uns nicht mit einem Bruder, sondern mit unserm Haupt, Jesus Christus, vergleichen – bringen uns oft tiefe Demütigungen; aber sie sind unerlässlich für unsere Selbstseelsorge, und wohl uns, wenn wir uns ihnen nicht entziehen! "Demütigungsstunden – Ewigkeitsstunden!" Man wird es bald unserem Wesen, Reden und Wirken anmerken, ob wir treu sind in der Selbstprüfung.
\par
Es ist allerdings in der Selbstseelsorge überaus wichtig, dass wir in solchen Stunden der Selbstprüfung uns nicht damit begnügen, nur zu erkennen, wo es fehlt, sondern dass wir uns tief reinigen lassen durch unseren großen Hohenpriester und aufs Neue die heiligen Priesterkleider anziehen, von denen 3.Mose 16,4 die Rede ist. Sie bestanden aus drei Stücken: aus den leinenen Beinkleidern, dem Lendengurt und der Kopfhaube. Für unseren Dienst ins Geistliche übersetzt: Tägliche Besprengung und Deckung unseres Fleisches- und Naturlebens durch das Blut Christi, Umgürtung der Lenden unserer Gefühlswelt und Unterstellung unserer Gedankenwelt unter die Zucht des Heiligen Geistes. Nur durch dieses Anziehen der heiligen Priesterkleider im Glaubensgebet hat die Selbstprüfung ihren wahren, bleibenden Wert.
\par
Als Zweites nennen wir die sogenannte \emph{"meditatio sacra"}, ein Wort, das sich nicht so leicht ins Deutsche übersetzen lässt: das geheiligte Nachsinnen, Nachdenken, im Herzen bewegen. Wir verstehen darunter nicht nur die stille Vorbereitung auf eine Bibelstunde. Wir haben es jetzt mit unserer eigenen Seele zu tun. Wir denken da vor allem an unsere intime Morgenandacht. Welch eine Stärkung bringt es unserer Seele, wenn wir in der Stille des Heiligtums vor der Bibel niederknien, das Wort Gottes einfach, zunächst noch ohne exegetische Durcharbeitung, in Gebet verwandeln, in unsere Seele einbetten, so dass es in unserer Seele liegen bleibt! Und dann tragen wir dies Wort in unserem Innern durch den ganzen Tag und bewegen es in freien Augenblicken. So bildet es einen Sammel- und Ruhepunkt und zugleich eine Bewahrungskraft für unsere Gedanken. Es wirkt unmittelbar auf uns. Es ernährt, durchwärmt und sättigt unsere Seele, und dann können wir es erfahren, wie ein solch persönlich erlebtes Wort oft gerade das Wort ist, das wir auch anderen Seelen im Auftrag Gottes bringen dürfen.
\par
Unter Meditation verstehen wir aber auch nicht nur das Sinnen über Gottes Wort, sondern auch das betende Nachdenken über unsere Erfahrungen. Wie wollen es doch nicht vergessen, dass die vielen Erfahrungen, die  wir machen, nicht an sich schon ein Reichtum unseres Lebens sind, sondern erst dann werden sie uns ein Segen, wenn wir diese Erfahrungen in uns bewegt und darüber nachgedacht, sie beleuchtet, geprüft, registriert haben und uns klar bewusst geworden sind, was sie uns zu sagen haben. Wie viele Erlebnisse, die uns ein Segen hätten sein können, blieben unfruchtbar und wertlos und rauschten wirkungslos vorüber, weil wir nicht darüber nachsannen in der Stille, weil wir sie nicht eingegliedert haben in unser Gesamtleben! Nur Verinnerlichung bringt wirkliche Bereicherung.
\par
Als drittes Mittel der Seelsorge in der Stille möchten wir nennen: \emph{Das Gebet} in seiner intimsten Form, nicht nur das Reden mit Gott, sondern das Stehenbleiben vor Gott zu innerster und innigster Herzenszwiesprache, das Lauschen auf seine Stimme und das Ausschütten des Herzens vor ihm. Da erst vernehmen wir das Rauschen des lebendigen Wassers und kann unsere Seele trinken aus göttlichem Quell. Die kurze, klassische Antwort, die jener junge Missionsarzt mir zurief, als ich ihn fragte, wie es ihm gehe: "Es geht mir gut, in dem Maß als ich bete", gilt auch uns für unser Seelenleben und Dienstleben.
\par
In solchem Gebet erleben wir das \emph{eine}, das für unser ganzes Leben von allerhöchster Bedeutung ist, das eine, das von Bileam bezeugt wird mit den Worten: "Dem die Augen aufgetan werden, wenn er niederkniet"\footnote{4.Mose 24,16}. Ja, Brüder, nur auf den Knien in der Stille empfangen wir den Ewigkeitsblick, diese wunderbare Fähigkeit der Seele, hindurchzuschauen durchs Sichtbare ins Unsichtbare, durch die Trübsal in die Herrlichkeit, durchs Verwesliche ins Unverwesliche. Erst durch diese tiefinnerliche Einstellung der Seele auf die Ewigkeit werden wir fähig, alles, was im Tageslauf an uns herantritt, sofort in der Ewigkeitsbeleuchtung, im Höhenlicht zu beurteilen. Es ist wunderbar, was für eine innere Befreiung und Erhebung wir erleben, sobald wir mit dem Ewigkeitsblick durchs Leben gehen. Da gibt es eine durchgreifende Umwertung aller Werte. Jetzt erst wird uns das Kleine klein und das Große groß. Dinge, die uns so stark in Anspruch nahmen, verlieren ihre Zwangskraft und fallen ab. Die Nöte, Kämpfe und Schwierigkeiten werden verklärt. Menschen, unter deren Wesen wir viel gelitten haben, erkennen wir jetzt in der Ewigkeitsbeleuchtung als Gehilfen Gottes zu unserer Erziehung. Unser Zusammenhang mit dem Irdischen wird gelockert; wir werden Ewigkeitsmenschen im Diesseitsleben. Unser Leben und Wirken erhält eine innere, klarere Höhenrichtung. So siegen wir über die Verzagtheit, Engherzigkeit und Abhängigkeit und dringen immer tiefer ein in die Freiheitsstellung der Gotteskinder. Es kommt eine heilige Großzügigkeit und Weitherzigkeit in unser Wesen. Das stärkt und entfaltet unser Seelenleben. Diesen Ewigkeitsblick erlangen wir aber nur in der Stille des Gebets. Er ist ein Gnadengeschenk unseres Gottes.
\hfill \break
\par
Lasst mich noch ein Letztes nennen in Bezug auf die Selbstseelsorge: Die \emph{geistliche und geistige Weiterarbeit}. Das entspricht einem ganz selbstverständlichen psychologischen Gesetz: Dass unsere Seele nur produktiv sein kann, wenn sie zugleich rezeptiv ist. Neue Gedanken entstehen nur durch geistige Empfängnis, durch Befruchtung bisher gewonnener und vorhandener Erkenntnisse. Wo diese Rezeptivität fehlt, verarmt, verkümmert allmählich das Innenleben und damit auch unser Dienst. Wir sind es den Seelen, denen wir dienen dürfen, schuldig, dass wir vor allem um die rechte Ernährung unserer eigenen Seele besorgt sind. Das kann ja in verschiedener Weise geschehen, vor allem selbstverständlich durch fortlaufendes Durcharbeiten ganzer Bücher der Bibel, durch ehrliches Sichauseinandersetzen mit biblischen Fragen und Zeitproblemen im Licht der Schrift. Solche Weiterarbeit, das ist heilsam für unsere Seele. Erkämpfte Erkenntnisse sind die wertvollsten. Wir wachsen an dem, was wir überwinden. – Aber nicht nur geistliche, sondern auch geistige Weiterarbeit ist nötig, und zwar auf verschiedenen Bildungsgebieten, wie in Psychologie, Geschichte, Literatur u. a. Auch das gehört zur Selbstseelsorge. Das erweitert unseren Horizont und hebt unsere Seele heraus aus den Schranken engen Denkens. Das befruchtet das ganze Innenleben, allerdings unter der Bedingung, dass alles in Beziehung gebracht wird zu unserem Lebenszentrum, zu Jesus Christus. "Alles ist euer, ihr aber seid Christi."\footnote{1.Kor 3,22-23} Ja, wir möchten bei diesem Punkt auch nicht unerwähnt lassen die Erholungsstunden durch Tätigkeitswechsel. Wenn die Mönche in den Klöstern hinter ihrer Arbeits- und Gebetszelle ein blühendes Gärtchen haben, in dem sie von Zeit zu Zeit Erfrischung suchen, so sollte auch jeder Seelsorger sein Gärtchen haben. Es kann ein wirklicher Garten, es kann Bienenpflege sein, Musik, Dichtkunst oder anderes mehr. Mein Klostergärtlein war längere Zeit das Atelier eines Kunstmalers, bei dem ich Montag früh als Sonntagserholung einige Stunden zubrachte mit Zeichnen und Malen. Köstliche Seelenerquickung! Gewiss  war für Paulus das Teppichweben nicht nur Broterwerb, sondern es war für diesen geistigen Arbeitsriesen wohl auch oft ein erwünschter und erfrischender Tätigkeitswechsel.
\par
Beim Hinweis auf diese verschiedenen Mittel und Wege der Selbstseelsorge ist es mir aber, als hörte ich bei so manchem Bruder den berechtigten Seufzer: Ach ja, gutgemeinte brüderliche Ratschläge, die an sich recht sind, aber woher Zeit nehmen zu solcher Innenpflege? Unmöglich!
\par
Gewiss, das wissen wir alle, dass es im Leben eines Reichgottesarbeiters, der seinen Dienst ernst nimmt, Zeiten geben kann, da das Arbeitsgedränge so stark ist, dass die Selbstseelsorge in dieser Weise einfach nicht zu ihrem Recht kommen kann. Wenn aber in solch außerordentlichen Zeiten die Arbeit uns wirklich vom \textsc{Herrn} zugewiesen ist, dann können wir die Gewissheit haben, dass er uns auch außergewöhnliche Stärkung zukommen lässt nach Jes 58,11: "Ich will deine Seele sättigen in der Dürre".
\par
Aber, Brüder, wir wollen uns doch ehrlich prüfen, ob wirklich immer nur das Übermaß der Arbeit und der Zeitmangel schuld daran sind, wenn wir zu wenig Zeit zur Stille finden. Könnte es nicht manchmal ein Mangel an richtiger, rationeller Zeiteinteilung sein? Wir arbeiten vielleicht nicht genügend planmäßig, oder wir verlieren Zeit durch geschäftigen Müßiggang. Man tut allerlei und vielerlei in Haus und Hof; aber wir arbeiten nicht genügend konzentriert, stehen nicht genügend unter Selbstkontrolle und Selbstzucht. Haben wir alle Augenblicke, die wir glaubten, nicht der Sammlung widmen zu können, wirklich zu ernster Arbeit verwendet? Haben wir uns  auch keine Arbeit aufgeladen, die nicht unbedingt zu unserer vom \textsc{Herrn} uns zugewiesenen Aufgabe gehört? Brüder, ob es vielleicht nicht auch bei manchem Reichgottesarbeiter gut wäre, wenn ab und zu ein Jethro käme wie damals zu Mose, als er unter Überlastung litt, und der ihm zurief: "Es ist nicht gut, was du tust! Du wirst müde und kraftlos zugleich. Das Geschäft ist dir zu schwer, und du kannst es nicht allein ausrichten." Lade ab; sage auch einmal nein! Wie leicht kann ein eifriger Bruder dahin kommen, dass er aus lauter Reichgotteseifer die Kosten nicht genügend überschlägt, immer wieder andere Arbeit sich aufladet, ohne zu prüfen, ob die eigene Seele es dann auch habe hinauszuführen. Nie sollte in der Reichgottesarbeit die Quantität auf Kosten der Qualität geschehen. Da ist weniger mehr. O, erbitten wir vom \textsc{Herrn} nicht nur den Geist der Kraft und der Liebe, sondern auch den Geist der Zucht, nicht am wenigsten zur richtigen Einteilung und Verwendung unserer Zeit! Da könnte gewiss noch manche Minute gewonnen werden für den Dienst hinter dem Vorhang.
\par
Endlich lasst mich noch auf das zweite Gebiet der Seelsorge an Seelsorgern hinweisen:
\section{Die brüderliche Seelsorge.}
Wir denken da zunächst an den intimeren Dienst von Bruder zu Bruder. Man sollte meinen, dass dieser brüderliche Dienst selbstverständlich unter uns reichlich gepflegt werde; aber Tatsache ist, dass sehr oft der Seelsorger, der so vielen dient, selbst keinen Seelsorger hat. Die einen haben kein Bedürfnis darnach; es sind selbständige, einfache, klare Naturen. Andere hätten das Bedürfnis wohl; aber sie wollen allein mit ihrem \textsc{Herrn} fertig werden. Noch andere hätten solchen Dienst sehr nötig; aber sie finden den Weg nicht zum Bruder, fürchten sich, ja, es fehlt ihnen vielleicht sogar das Vertrauen.
\par
Da ist sicherlich ein Mangel vorhanden. Das werden die am besten bezeugen können, die den reichen Segen solch brüderlicher Seelsorge selbst kennen und genießen. Ich las unlängst ein ausgezeichnetes katholisches Buch, das auch uns Evangelischen etwas zu sagen hat: "Des Priesters Heiligung", Erwägungen für Seelsorger von Prälat Dr. Jakob Schmitt. Beim Lesen dieses Buches fiel mir auf, welch großen Wert die katholische Kirche gerade auf die Seelenführung der jüngeren Priester legt, wie sie dafür sorgt, dass jeder einen Beichtvater hat, der sogar verpflichtet ist zur \emph{"correctio fraterna"}, zu brüderlicher Zurechtweisung, und zwar soll dies geschehen in möglichst liebevoller, teilnehmender, heilig-ernster Weise.
\par
Brüder, ob nicht auch wir in unserem Kreis die brüderliche Seelsorge mit mehr Eifer und Treue pflegen sollten? Gewiss lässt sie sich nicht befehlen. Es darf durchaus nichts "Gemachtes" sein. Seelsorge gedeiht nur auf der Grundlage des Vertrauens, und Vertrauen lässt sich nicht befehlen. Aber wenn die Erkenntnis der Wichtigkeit und des Segens solchen Dienstes vorhanden ist und ein liebevolles Entgegenkommen, eine seelsorgerliche Bereitwilligkeit, eine brüderliche Handbietung spürbar ist, dann lässt sich manches weiter ausbauen. Wir sind ja auch als begnadigte Gottesknechte doch noch Menschen von Fleisch und Blut, Menschen, die noch nicht fertig sind mit der Durchheiligung ihres Lebens, die auch ihre inneren Konflikte haben. Ja, wir wissen es nur zu gut, wie Satan es ganz besonders auf uns abgesehen hat. Ist es da nicht ganz natürlich, dass es auch für uns Stunden innerer Großnot geben kann, wo uns brüderlich-seelsorgerliche Handreichung ein dringendes Bedürfnis wäre? Ja, lasst mich ganz offen einen Punkt berühren! Unser Seelsorgerdienst bringt es mit sich, dass wir auch Bekenntnisse hören müssen, in denen die ganze dämonische Macht der Erotik uns entgegentritt. Wie leicht können da Funken hereinspringen und Zündstoff in der eigenen Natur und Gedankenwelt zum Aufflammen bringen! Wie mancher Bruder, der vielleicht durch Vererbung unter sinnlicher Veranlagung litt, ist auf diesem Gebiet in schwere innere Not geraten, hat die Glut nicht mehr löschen können, und sie hat weiter um sich gegriffen. Wäre er zu einem Bruder gegangen und hätte sich offen ausgesprochen, hätte ihm geholfen werden können. Aber aus falscher Scheu unterließ er es, schwieg, litt, verzehrte sich und musste das Wort erfahren: "Da ich es wollte verschweigen, verschmachteten meine Gebeine!"\footnote{Ps 32,3} Wie mancher hat Schiffbruch erlitten und ist dienstuntauglich geworden! Der \textsc{Herr} schenke uns Augensalbe, dass wir's merken, wenn ein Bruder in Not ist! Er stärke und vertiefe unser gegenseitiges Vertrauen! Brücken müssen von beiden Ufern gebaut werden; so auch die Vertrauensbrücken.
\par
Aber es gibt noch eine andere Art brüderlicher Seelsorge, die vielfach ganz unbewusst, ganz unhörbar vor sich geht. Ich denke da an das Wort Jesu: "Ich heilige mich selbst für sie"\footnote{Joh 17,19}. Wir wissen ja, wie dies Wort bei Jesus zu verstehen ist, wie diese Selbstheiligung als Selbsthingabe im Opfertod am Kreuz die Quelle unseres Heils geworden ist. Wenn auch in anderem Sinne, so kann dies Wort uns doch als Richtlinie dienen für unsere brüderliche Seelsorge. Wir wirken ja bekanntlich aufeinander viel mehr durch das, was wir sind, als durch das, was wir reden. Von jedem von uns geht eine ganz bestimmte Geistesatmosphäre aus, ein Seelengeruch, entweder ein Ichgeruch oder ein Jesusgeruch. Und gerade durch diese unbewusste und unmittelbare Seelenausstrahlung wirken wir am stärksten aufeinander. Stehen wir ernstlich in der Heiligung, sind wir Menschen, in denen der Heilige Geist jedes Lebensgebiet mehr und mehr durchdringen kann, dann geht ein Geisteseinfluss von uns aus, der zu einer starken Hilfe werden kann für unsere Brüder. Ich übertreibe nicht, sondern rede aus eigenster Erfahrung, wenn ich sage, dass die ganze Dienstauffassung und innere Entwicklung eines jungen Bruders auf Jahre hinaus bestimmt werden kann durch das Wesen und Wirken des älteren Bruders, mit dem er zusammenarbeitet. Wie mancher junge Reichgottesarbeiter hat da tiefe und segensreiche Impulse für seinen künftigen Dienst empfangen! Wie mancher aber hat die Ideale, mit denen er in den Dienst trat, tiefer gestellt, als er sah, wie der ältere Bruder seine Arbeit leichter und äußerlicher auffasste und im Privatleben sich gehen ließ! Auch da wollen wir das Wort Inspektor Rappards beherzigen: "Brüder, sorgt dafür, dass euer Dienstleben und Privatleben stets aus einem Guss sei!" Vergessen wir nicht, dass unsere Heiligung nicht eine rein persönliche Sache ist, sondern dass immer ein Einfluss von uns ausgeht auf unsere Nächsten: ein Segen oder ein Fluch! Lasst uns nachjagen der Heiligung! Dann werden wir unbewusst manchem Bruder seelsorgerlich dienen können und werden auf diese Weise dem Wort Jesu nachleben: "Ich heilige mich selbst – für ihn!"
\hfill \break
\par
Eine zweite Gelegenheit gegenseitiger Seelsorge kann der weitere Kreis unserer brüderlichen Gemeinschaft uns bieten. Ich denke jetzt an unsere \emph{Konferenzen}. Ob wir nicht auch da etwas lernen könnten von unseren katholischen Amtsbrüdern? Es besteht unter ihnen eine freie Vereinigung, die sogenannte \emph{"unio apostolica"}, die einen seelsorgerlichen Zweck verfolgt und gewiss viel Segen stiftet. Die Priester, die ihr beitreten, verpflichten sich, täglich gewisse seelsorgerliche Vorschriften zu befolgen und sich gegenseitig Rechenschaft darüber zu geben. Unter anderem verpflichten sie sich, jeden Tag mindestens eine halbe Stunde der stillen, betenden Sammlung zu widmen, eine Viertelstunde dem theologischen Studium und eine Viertelstunde der Selbstprüfung und der geistlichen Erbauungslektüre. Täglich tragen sie in ein Büchlein ein, ob sie diesen Vorschriften treu waren, und was sie dabei erlebt haben. Jeden Monat senden sie das Büchlein ihrem Beichtvater oder Provinzialvikar, der dann seelsorgerliche Mahnungen daran knüpft. Es mag sein, dass wir vielleicht ein überlegenes Lächeln haben für solchen katholischen Seelenzwang. Das brauchen wir Evangelischen nicht, und gewiss wäre es töricht, die Seelenübung einfach in der gleichen Form nachzuahmen. Aber – sind wir nicht manchmal in Gefahr, das Kind samt dem Bade auszuschütten und in das Gegenteil zu verfallen, in eine Unterschätzung alles Zwanges, aller Selbstzucht und Selbstkontrolle? Hat Vinet nicht doch recht, wenn er sagt: "Freiheit ist die Gebundenheit, die man sich selber gibt?" Ja, keine Freiheit ohne Zucht!
\par
Könnten wir nicht unseren Konferenzen auch etwas mehr den Charakter einer \emph{"unio apostolica"} geben, sie etwas mehr noch in das Zeichen der Seelsorge an Seelsorgern stellen? Es scheint mir, dass manchmal unsere Konferenzen allzu viel im Zeichen der Exegese, des Aufsatzes, des Referates, der äußeren Geschäfte stehen und ihnen dadurch der Ton brüderlicher Seelsorge fehlt. Ich habe mich herzlich gefreut, als kürzlich an einer unserer Konferenzen ein Bruder während der Besprechung der Exegese, anknüpfend an den vorliegenden Bibelabschnitt, freimütig das Bekenntnis aussprach: "Ach, ich leide öfters an Verzagtheit! Wo liegt wohl die Ursache? Erlebt ihr Ähnliches?" Dadurch wurde die Aussprache sofort in eine andere Bahn gelenkt. Sein Bekenntnis rief anderen Bekenntnissen. Es kam eine persönliche Wärme in die Aussprache, die dann weiterströmte ins gemeinsame Gebet. Versuchen wir, in unseren Konferenzen der Seelsorge auch Raum zu geben, uns schon vorher darauf einzustellen und dafür zu beten! So werden wir uns gegenseitig auch auf diesem Gebiet immer besser seelsorgerlich dienen können als eine evangelische \emph{"unio apostolica"}.
\par
Ein weiterer Kreis seelsorgerlicher Hilfskräfte sind \emph{die Erfahrungen, die unser Dienst uns bringt}. Es können demütigende Erfahrungen sein, Misserfolge, schmerzende Niederlagen, durch die der \textsc{Herr} uns im eigenen Leben etwas aufdecken möchte. Oder es sind Beobachtungen, die wir bei unseren Mitmenschen gemacht haben. Oder es sind Begegnungen mit Menschen, vielleicht unangenehmen Menschen voller Widersprüche und Rechthaberei, die uns den Dienst schwer machen, die uns aber gerade durch ihr Wesen Gelegenheit geben sollen zur Selbstprüfung und zum Wachstum in der Sanftmut. Oder es sind reife, gegründete Christen, die weiter voran sind in der Heiligung als wir, die uns als Ansporn dienen sollen, wie dies zu Hiskias Zeiten bei den Leviten der Fall war, von denen es heißt: "Die Leviten waren ernstlicher darauf bedacht, sich zu heiligen als die Priester"\footnote{2.Chr 29,34}. Sollen uns alle diese Erfahrungen seelsorgerlich dienen können, dann kommt es allerdings darauf an, dass wir sie auf uns wirken lassen, sie innerlich verarbeiten, uns darunter stellen, uns beugen und davon lernen. Dann wachsen wir innerlich durch unseren Dienst.
\par
Lasst mich endlich noch das reichste und intimste Gebiet seelsorgerlicher Hilfe nennen: \emph{Unser Eheleben}, der Dienst unserer Gattin! Sie soll unsere Gehilfin sein, also auch unsere Seelsorgerin. Und wer könnte diesen Dienst besser erfüllen, als gerade sie, die uns am gründlichsten kennt, am innigsten liebt, und die unser volles Vertrauen hat? Ja, unsere eheliche Liebe sollte vor allem eine seelsorgerliche Liebe sein. Wir sollen nicht nur Anteil nehmen am Seelenleben unserer Frau und ihr zu dienen suchen, sondern wir sollen ihr auch Anteil geben an unserem Seelenleben, sie hineinschauen lassen in das, was uns bewegt und Kampfesnot bereitet. So können wir uns gegenseitig seelsorgerlich dienen, können – um mit dem feinen Wort von Niebergall zu sprechen – "einander in die Höhe lieben". Ja, wachen wir eifersüchtig darüber, dass uns nie der Dienst, die Arbeit, das Gedränge im Vorhof, die Tür zum Heiligtum der Ehe zuschließt! Wo Mann und Frau in liebesstarker Wahrhaftigkeit einander seelsorgerlich helfen, da erweisen sie sich gegenseitig den höchsten Liebesdienst.
\begin{center}
* * *
\end{center}
\par
Und nun, liebe Brüder, lasst uns mit neuem Ernst und helliger Entschiedenheit den Zuruf des Apostels beherzigen und verwirklichen: "Habt acht auf euch selbst!"\footnote{Apg 20,28} Lasst uns mit Sorgfalt und mit Glaubensmut die Mittel und Wege gebrauchen, die der \textsc{Herr} uns zu unserer Innenpflege zur Verfügung stellt! Je treuer wir sind nach innen, unserer eigenen Seele gegenüber, desto fruchtbarer wird auch unser Dienst sein an der Herde, die der \textsc{Herr} uns anvertraut. Ja, er selbst, unser großer Hohepriester, Jesus Christus, der sich unserer Seele so herzlich angenommen hat, dass sie nicht verdürbe und sie erkauft hat durch sein teures Blut, \emph{er selbst gebe uns Kraft nach dem Reichtum seiner Herrlichkeit, stark zu werden am inwendigen Menschen durch seinen Geist!}\footnote{Eph 3,16}
\end{document}
